% Created 2022-12-02 Fri 11:57
% Intended LaTeX compiler: xelatex
\documentclass[11pt]{article}
\usepackage[mathletters]{ucs}
\usepackage[mathletters]{ucs}
\usepackage{color}
\usepackage{listings}
\usepackage[english]{babel}
\usepackage[T1]{fontenc}
\usepackage[a4paper,bindingoffset=0.2in,left=1in,right=1in,top=1in,bottom=1in,footskip=.25in]{geometry}
\usepackage[dvipsnames]{xcolor}
\usepackage{fontspec}
\usepackage[math-style=french]{unicode-math}
\usepackage{mathtools}
\setmathfont[math-style=upright]{DejaVu Sans Mono}
\setmonofont[Color=blue]{Ubuntu Mono}
\newfontfamily{\mm}[Color=red]{DejaVu Sans Mono}
\setmainfont[BoldFont=EB Garamond,BoldFeatures={Color=ff0000}]{EB Garamond}
\newcommand{\hookuparrow}{\mathrel{\rotatebox[origin=c]{90}{$\hookrightarrow$}}}
\usepackage{fix-abstract}
\definecolor{pale}{HTML}{fffff8}
\definecolor{orgone}{HTML}{83a598}
\definecolor{orgtwo}{HTML}{fabd2f}
\definecolor{orgthree}{HTML}{d3869b}
\definecolor{orgfour}{HTML}{fb4933}
\definecolor{orgfive}{HTML}{b8bb26}
\definecolor{gruvbg}{HTML}{1d2021}
\newenvironment*{emptyenv}{}{}
\usepackage{sectsty}
\sectionfont{\normalfont\color{red}\selectfont}
\subsectionfont{\normalfont\selectfont}
\paragraphfont{\normalfont\selectfont}
\subsubsectionfont{\normalfont\selectfont\color{black!50}}
\author{Noorah Alhasan}
\date{\today}
\title{Hyperreal Enterprises: Workshop Manual}
\begin{document}

\maketitle
\begin{abstract}
\noindent This document is a linear treatment of the PLACARD workshop.
\end{abstract}

\setcounter{tocdepth}{2}
\tableofcontents
\section{PLACARD workshop introduction}
\label{b7b42aa2-c57c-4bcc-bc45-be9b63972be7}
\subsection{Instructions}
\label{sec:orgb90d76f}

\subsubsection{Introductions}
\label{sec:orgfc4c538}
\begin{itemize}
\item Bristol Pilot: Abby presented a 10 minute slide show about the ``Public Space and Public Health''
\end{itemize}
\subsubsection{Meet and greet with participants}
\label{sec:org1cc3ed0}
\begin{itemize}
\item Bristol Pilot: minimal, people said hi
\end{itemize}
\subsubsection{{\bfseries\sffamily TODO} ice-breaker activity}
\label{sec:org84be6f0}
\begin{itemize}
\item \emph{Could be based on an in-advance ask, like, email a photo?}
\item Maybe better to have it be something that can happen at a coffee break.
\end{itemize}
\subsubsection{{\bfseries\sffamily TODO} Go over the general rules and norms}
\label{sec:org05fe842}
\begin{itemize}
\item Bristol Pilot: Joe presented an outline of what was going on on the day
\end{itemize}
\subsubsection{Make sure we have a clear terminology/taxonomy per phase, not only for patterns and roles but the mundane terms that we use}
\label{sec:org502a528}
\subsubsection{Explain theme of the workshop}
\label{sec:orgcffff23}
\begin{itemize}
\item NB. different people/groups might go in different directions from this point.
\end{itemize}
\subsubsection{Explain the process}
\label{sec:orga40a5c2}
\begin{itemize}
\item ``Discuss the theme, discuss the scenarios for development, zoom in on those, flesh them out.''
\item ``Find new ‘design patterns’ that enable them to solve new problem.''
\end{itemize}
\begin{enumerate}
\item Explain CLA (also good for us as well) \textbf{and/or} present the framework with some tangible interactives, like cards, so they don’t get bored
\label{sec:org0f21535}
\begin{itemize}
\item We can explain how this will be applied, e.g., via an \textbf{example}
\item Diagrams can convey information quickly (e.g., iceberg)
\item Roles \& patterns are visual and memorable
\item How will it be used in this phase: decent to the myth
\item Be able to access the rules quickly (cf. ``Settlers of Catan'')
\end{itemize}
\end{enumerate}

\subsection{Workshop management checklist}
\label{e28fb669-45a6-4916-b56b-a3afd6238d4f}
\begin{itemize}
\item[{$\square$}] Make sure everyone is on the same page
\item[{$\square$}] Develop a clear taxonomy of the fundamental terms
\item[{$\square$}] Get contracts \& deliverables sorted
\item[{$\square$}] Know how much flexibility there is (what’s possible, and what’s possible when)
\item[{$\square$}] Reasonable expectations about where this is going
\item[{$\square$}] Reasonable expectations how much effort it needs, and what’s available
\item[{$\square$}] Make sure we have a protocol about this
\item[{$\square$}] Make good commits for this document!
\end{itemize}
\subsection{Workshop frontstage organisation}
\label{2a01f142-31c7-4e86-ae10-e14e85b4dda9}
\begin{enumerate}
\item \textbf{Send invitation e-mails at least 4 weeks in advance}
\begin{itemize}
\item Send reminder \textbf{3 weeks} in advance
\item Get confirmation as early as possible, cut-off date \textbf{2 weeks} in advance
\item \textbf{Follow best practices for human participation activities:}
\begin{enumerate}
\item Include a \emph{brief} blurb about the workshop (\textbf{including what the commitment} is, e.g., full-day) and what their participation will accomplish
\item Include the minimum level of technology knowledge (can they use a PC, browse the web, bring a phone, typing into a website, etc.?)
\item Include the level of security/privacy of participation and ask for consent
\item Include date, time, duration, and location (if it's TBD make sure to include this info)
\item Include \uline{whether} their participation will be monetarily compensated
\item Thank them for their time and consideration
\item Introduce a warm-up activity like \hyperref[615846a2-1795-40b4-8dfb-3e12923fccc0]{Dérive Comix}
\end{enumerate}
\end{itemize}
\item \textbf{Once we have confirmations, send another e-mail.}
\begin{enumerate}
\item Include a somewhat more detailed outline of the workshop, reminding them what it’s about, and reminding them what their participation will accomplish
\item Include timeline of the workshop (\textbf{full day timings here}):
\begin{enumerate}
\item Introductions, overall workshop overview (30 minutes)
\item Explaining Phase 1 (15 - 30 minutes)
\item Phase 1 (1.5 - 2 hours)
\item Lunch break (1 hour)
\item Explaining Phase 2 (30 - 45 minutes)
\item Phase 2 (1.5 - 2 hours)
\item Concluding remarks
\end{enumerate}
\item Include pre-workshop activity or any other asks
\item Thank them for their time and consideration
\end{enumerate}
\item \textbf{Run the workshop and try to ensure that folks have a good time.}
\item \textbf{After the workshop, follow up appropriately.}
\end{enumerate}
\subsection{Workshop backstage organisation}
\label{781d52fa-71a9-4c90-b4f6-9b0dd4244c33}
Be sure to sort these as-needed:

\begin{itemize}
\item Methods, needed at the appropriate level of rigour
\begin{itemize}
\item \textbf{Decide: In-person vs remote vs hybrid?}
\item Technologies for the workshop
\item Facilitation
\item Feedback platform
\item Technologies for after the workshop
\end{itemize}
\end{itemize}

We need every component to fit well with the methods.

\begin{enumerate}
\item Criterion: The methods need to be tested and familiar to facilitators
\label{sec:orgf45df24}
\end{enumerate}
\subsection{Prior to the workshop}
\label{9accd402-6d28-4ee2-ac35-44b4fe682d53}
\subsubsection{Instructions}
\label{sec:org9a4ba9e}

Follow instructions at
\begin{itemize}
\item \hyperref[e28fb669-45a6-4916-b56b-a3afd6238d4f]{Workshop management checklist},
\item \hyperref[781d52fa-71a9-4c90-b4f6-9b0dd4244c33]{Workshop backstage organisation},
\item \hyperref[2a01f142-31c7-4e86-ae10-e14e85b4dda9]{Workshop frontstage organisation}.
\end{itemize}

\subsection{During the workshop}
\label{c8823bc4-d08e-4486-9841-c914bba9977e}
\subsubsection{Instructions}
\label{sec:org7d32676}

Introduction: Follow instructions at \hyperref[b7b42aa2-c57c-4bcc-bc45-be9b63972be7]{PLACARD workshop introduction}.

Phase I: Follow instructions at \hyperref[95072d03-1359-4863-bad1-651191eb2f38]{Participatory Scenario Planning}.

Phase II. Follow instructions at \hyperref[85fefbc1-ca57-46fa-a8b2-154821a56c75]{Play to Anticipate the Future}.
\subsection{After the workshop}
\label{3d0acf49-0c87-4aaa-94b3-84e5d926d58d}
\subsubsection{Instructions}
\label{sec:org58d62b1}

Phase III. Follow instructions at \hyperref[92e18906-d0e6-4e73-a9cf-fbdad931f3cf]{Roadmap}.
\section{Open Future Design}
\label{66d6f9a0-c5ab-480d-8010-5c645aeadc17}
\subsection{Summary}
\label{sec:orgb6310a4}

\textbf{*Context} People need to coordinate, plan, and maintain cohesion. \textbf{If} a
culture can develop based on shared learning BUT there is no reliable
oracle that can tell us what to expect; \textbf{Then} use design pattern
methods to articulate multiple futures. This work can be guided by
further patterns, e.g., to develop languages of:

\begin{itemize}
\item \emph{future scenarios} →  \hyperref[95072d03-1359-4863-bad1-651191eb2f38]{Participatory Scenario Planning}
\end{itemize}

\begin{itemize}
\item \emph{roles} →  \hyperref[85fefbc1-ca57-46fa-a8b2-154821a56c75]{Play to Anticipate the Future}
\end{itemize}

\begin{itemize}
\item \emph{projects} →  \hyperref[92e18906-d0e6-4e73-a9cf-fbdad931f3cf]{Roadmap}
\end{itemize}

\subsection{Instructions}
\label{sec:orgd8eabf1}

\begin{enumerate}
\item Follow instructions at \hyperref[9accd402-6d28-4ee2-ac35-44b4fe682d53]{Prior to the workshop}.
\item Follow instructions at \hyperref[c8823bc4-d08e-4486-9841-c914bba9977e]{During the workshop}.
\item Follow instructions at \hyperref[3d0acf49-0c87-4aaa-94b3-84e5d926d58d]{After the workshop}.
\end{enumerate}

\subsubsection{Documentation}
\label{sec:org4bcc1f6}

This is the entry point pattern for the Open Future Design Pattern
Language.  Instructions for building all of the cards in a printable
format are as follow.  The command is to be run in the directory
\texttt{./erg/pattern-cards/}.

\begin{verbatim}
pdflatex patterns-reboot-figures.tex && \
 pdflatex patterns-reboot-figure-main.tex && \
 pdflatex patterns-reboot-roles.tex && \
 pdflatex patterns-reboot-roles-2.tex && \
 pdflatex patterns-reboot-roles-3.tex && \
 pdflatex patterns-reboot-roles-4.tex && \
 pdflatex scenario-planning-patterns.tex && \
 pdflatex patterns-reboot-par.tex && \
pdfjam \
patterns-reboot-figures.pdf \
patterns-reboot-figure-main.pdf \
patterns-reboot-roles.pdf \
patterns-reboot-roles-2.pdf \
patterns-reboot-roles-3.pdf \
patterns-reboot-roles-4.pdf \
scenario-planning-patterns.pdf \
patterns-reboot-par.pdf \
--outfile open-future-design-patterns.pdf --paper a5paper
\end{verbatim}
\section{Participatory Scenario Planning}
\label{95072d03-1359-4863-bad1-651191eb2f38}
\subsection{Summary}
\label{sec:org5a81d99}

\textbf{*Context} you want to plan for possible future scenarios. \textbf{If} you have
an interested group BUT no ``expert'' has all the answers; \textbf{Then} pool the
collected expertise of the affected communities.

\subsection{Instructions}
\label{sec:org118a0fd}

\subsubsection{Overall process}
\label{sec:org748c1ef}

\begin{itemize}
\item The initial input are participants' individual reflections on the workshop theme (\hyperref[615846a2-1795-40b4-8dfb-3e12923fccc0]{Dérive Comix}).
\item We want to make one or more big mind-maps (\hyperref[407beae8-ab2f-4340-9552-211d3b92ede6]{Meaning Map}). This can be structured by further activities:
\begin{itemize}
\item \hyperref[a853be79-85c1-4ffa-9750-459192c539e8]{Find the dots}
\item \hyperref[0ef4b185-513f-40c2-b884-6213601bbe09]{Advice from a Caterpillar}
\end{itemize}
\item These can be improved by coming up with dimensions in which things can optimise in some direction (good/bad) ([BROKEN LINK: 105e0ad7-ada7-4cee-b2c6-a68d08096159])
\item Once we have dimensions, we can describe scenarios that merge the themes and dimensions, outlining possible directions of development ([BROKEN LINK: 7357a42a-9691-4669-92c3-895d9061dda5])
\item The maps can be enriched by bringing in areas of expertise (\hyperref[bf8791b5-e50b-4666-bc01-286e279a5971]{Reinfuse Expertise})
\item Post Phase 1 immediate feedback; check on participants (like a mini-PAR).
\end{itemize}

\subsubsection{Materials}
\label{sec:orgadc9bfb}
\begin{itemize}
\item Big sheets of paper
\end{itemize}

\subsubsection{Intended outcomes}
\label{sec:org4fa3470}
\begin{itemize}
\item Merge everything into one big shared possibility space
\item Identify core issues
\end{itemize}

\subsubsection{Responsibilities of facilitators}
\label{sec:orga892e45}

\begin{itemize}
\item Move things along, e.g., so that we identify and elaborate the blockers, but don’t get stuck on them
\item Merge and evolve the mindmaps both within groups and across groups
\item Identify [BROKEN LINK: d7c5081f-cc76-4893-9daa-ff13b9bf1ae2].
\end{itemize}

\subsection{Phase I Feedback}
\label{sec:orgcd3764e}

\subsubsection{Improve processing of input}
\label{sec:orgc1aa67e}
\begin{itemize}
\item Consider \textbf{centralising the data}, since a lot of stuff is currently buried.
\item Explain where the notes are, how to find the \LaTeX{} files, git repos, \&c.
\item This follows patterns of the EmacsConf organisation, e.g., minimal commits, logs, etc.
\end{itemize}

\subsubsection{How to merge multiple mindmaps?}
\label{sec:org5ef4ea9}
\begin{itemize}
\item \textbf{Merging} might be easier to do with software, without them, you can get a spaghetti-fest; distilling, by asking ``What are the key nodes?''; it feels like this is particularly important.
\item Building something that distills the info
\end{itemize}

\subsubsection{Phase I Evolution}
\label{sec:orga7380f4}

\begin{itemize}
\item JC: Initial translation of materials into Org Roam notes (21/11/2022)
\end{itemize}
\subsection{Dérive Comix}
\label{615846a2-1795-40b4-8dfb-3e12923fccc0}
\subsubsection{Summary}
\label{sec:org72ecb42}

\textbf{*Context} you want to develop some future scenarios to explore with a
group. \textbf{If} you have an group BUT everyone has their own experiences;
\textbf{*Then} Go for a walk or just look out the window wherever you, and
document what you see. Follow up by preparing your materials to share in
a succinct fashion, e.g., as photos, a screenshot, slides, sketches, a
zine, a map, or some PostIt notes.

\subsubsection{Instructions}
\label{sec:orgaaafe98}

Walk for one hour around your neighbourhood.  Address some or all of
the following questions, possibly documenting them with photos, text,
or video clips.

What are you observing (sight, sound, smell)?  What are the obvious
things?  What are the sites of meaning, e.g., a bowl that is more than
just ‘a bowl’?  Where is meaning made unclear or fragmented?  What are
you experiencing (feelings, thoughts, first impressions)?  How have
things changed?  (It’s OK to get lost, but if you’re feeling lost when
reading these instructions, you may want to read this short intro to
the \href{https://www.publicstreet.org/derive}{dérive}.)

Follow up by preparing your materials to share in a succinct fashion,
e.g., as slides, sketches, a zine, a map, or some PostIt notes.
\subsection{Meaning Map}
\label{407beae8-ab2f-4340-9552-211d3b92ede6}
\subsubsection{Summary}
\label{sec:org28f58fb}

\textbf{*Context} We have collected images describing people's worlds (see
\hyperref[615846a2-1795-40b4-8dfb-3e12923fccc0]{Dérive Comix}). \textbf{If} you want to distill shared meaning BUT everyone has
their own experience; \textbf{Then} talk together about the problems and
opportunities that everyone sees. Maybe some of these will cluster
together, or maybe everyone will have their own different perspective:
that's OK. You can use these different viewpoints to get everyone on
the same map.

\subsubsection{Instructions}
\label{sec:org679c057}

Return to your small groups and bring together the themes you
identified earlier.  Informed by your reflections, work together with
the group to arrange the information on a map.  Notice that since
people navigated different physical locations, your ‘map’ is likely to
be somewhat abstract.  Where it makes sense, the map should record
different perspectives from people in the group.  For example, the
older people might perceive the place they explored to be a village,
while younger people perceive it to be a settlement on the outskirts
of town.  You might have different perspectives on what’s missing.
Try to articulate such complexities.
\subsection{Reinfuse Expertise}
\label{bf8791b5-e50b-4666-bc01-286e279a5971}
\subsubsection{Summary}
\label{sec:orga340b4e}

\textbf{*Context} a group wants to build a \hyperref[407beae8-ab2f-4340-9552-211d3b92ede6]{Meaning Map}. \textbf{If} everyone has
experience as a citizen BUT they also have expertise; \textbf{Then} begin by
removing expertise to get everyone on the same page, and subsequently
reinfuse expertise to enable richer and more complex thinking.
\section{Play to Anticipate the Future}
\label{85fefbc1-ca57-46fa-a8b2-154821a56c75}
\subsection{Summary}
\label{sec:orga288b15}

\textbf{*Context} you want to have fun with friends, colleagues or
acquaintances. \textbf{If} you want to explore possible futures BUT time travel
does not exist and you don't know what to expect; \textbf{Then} play a game
that lets you experience a plausible future scenario together.

\subsection{Instructions}
\label{sec:org8b6051f}

\subsubsection{Overall process}
\label{sec:org6528878}

\begin{enumerate}
\item Rapid training:
\label{sec:orgabef6a4}

\begin{itemize}
\item \textbf{Explain the game}:
\begin{itemize}
\item We should explain the game by trying it a few times.
\end{itemize}
\item \textbf{Explain the roles}:
\begin{itemize}
\item We need to convey that each role convey a little bit of a design pattern, or multiple of those.
\item “Why can’t I be myself?” - Yes you can, you just get a different colored scarf
\item How should each participant use these roles?
\end{itemize}
\item In short, what does each role represent?
\begin{itemize}
\item ``‘However’, ‘because’, ‘therefore’, ‘specifically’''.
\end{itemize}
\item What are the rules of the game?
\begin{itemize}
\item\relax [Need more time to re-do the discussion multiple times to optimize this.]
\end{itemize}
\end{itemize}

\item Scenario exploration:
\label{sec:orga2c0f30}

\begin{itemize}
\item Enrich the scenarios using the familiar pattern from \hyperref[f447153f-7ff5-449d-bb08-67f579dda53f]{Dérive Comix Part 2}.
\item Develop \hyperref[7c0dce3b-d5ea-4712-a771-6ff26f143686]{A path forward}.  Each role as a part to play in doing this:
\begin{itemize}
\item \hyperref[e38d2006-bcf7-494b-bd51-d8932b1ed0cd]{Back to reality} (Analyst)
\item \hyperref[34be214c-5885-4794-b93c-84e49ddad18b]{Connections from Kafka} (Kaijū Communicator)
\item \hyperref[ed238393-a7e4-4a0d-9eb2-3d6ab745c170]{New patterns} (Designer)
\item \hyperref[baa168fb-37a0-4144-ab16-d4962728ea9c]{Project Action Preview} (Historian)
\end{itemize}
\item If there's time, \hyperref[092e4fe4-ee4f-494d-8776-c5f1389e8dc0]{Repeat Phase II} for extra practice.
\item \hyperref[848c8c3d-cde3-48b4-9dae-23eca4db440d]{Share back}
\item \hyperref[f5a1bc15-5abb-44d6-8f7a-e254974c9002]{Project Action Review}
\end{itemize}
\end{enumerate}

\subsubsection{Materials}
\label{sec:orgad6d242}
\begin{itemize}
\item Need ``another big sheet of paper'' to elaborate what that thing is like, e.g., what is it like for you, what is it like for me.
\end{itemize}

\subsubsection{Intended outcome}
\label{sec:org7a90b87}
\begin{itemize}
\item Reverse CLA process to return from Myth to Litany.
\begin{itemize}
\item ‘Produce new headlines.’
\end{itemize}
\end{itemize}

\subsection{Phase II Feedback}
\label{sec:orgb0c278a}

\subsubsection{Strategy: consider using Org Roam intelligently}
\label{sec:org5f47982}

\begin{itemize}
\item We’d stopped using it as we originally intended, and just had meeting notes
\item Leo’s happy to create a slip-box following the patterns of Noorah’s agenda \& create an operational manual
\item This will be a ‘moderated’ shared slip box; we can have all the data so far, can create notes, read things, etc.
\item \textbf{Method for maintaining structure} can be taught later after we have the contents
\end{itemize}

\subsubsection{Phase II Evolution}
\label{sec:org50e0b06}

\begin{itemize}
\item Could we view ``evolution'' inside the diagram?
\item We do have ways to track \& see how things have changed; it'd be good to upgrade the interface
\item If the tools aren't there yet, we can mimic the tools.
\item Anyway, using Org Roam again will help us see where the \emph{feature evolution} should be
\end{itemize}
\subsection{Kaijū Communicator}
\label{a0796d9e-664b-46fa-bb37-7f6a6fc15584}
\subsubsection{Summary}
\label{sec:org9894ee9}

\textbf{*Context} When developing a vision of the future. \textbf{If} people start to
agree BUT no one challenges what's going on, solutions become brittle;
\textbf{*Then} use words like ``\emph{however}'' to challenge proposals and highlight
conflicts.
\subsection{Historian}
\label{57d46961-a056-435e-85d2-27ab6e0de7f6}
\subsubsection{Summary}
\label{sec:org2436bdb}

\textbf{*Context} When developing a vision of the future. \textbf{If} people start to
agree BUT no one connects it with local history and concrete actions,
then work bogs down; \textbf{Then} use words like ``\emph{specifically}'' to connect
abstract problems and solutions to specific actions.
\subsection{Analyst}
\label{5826c7d9-8962-433d-83c5-27a5196908ea}
\subsubsection{Summary}
\label{sec:orgc01c7c1}

\textbf{*Context} When developing a vision of the future. \textbf{If} people start to
form a solution BUT no one connects it with the complex reasons why that
solution is likely to work, then it's likely to be fragile; \textbf{Then} use
words like ``\emph{because}'' to describe the complex reasons that the solution
is likely to work.
\subsection{Designer}
\label{48a1d6a3-800d-46bd-8a4a-0d3414ecf150}
\subsubsection{Summary}
\label{sec:org68ffc62}

\textbf{*Context} When developing a vision of the future. \textbf{If} people start to
form a solution BUT we don't connect it with our existing knowledge,
then it's likely to be fragile; \textbf{Then} use words like ``\emph{therefore}'' to
describe the solution in terms of other known solutions.
\subsection{Find the dots}
\label{a853be79-85c1-4ffa-9750-459192c539e8}
\subsubsection{Instructions}
\label{sec:orgeaa3865}

Join together with other workshop participants in small groups to
share your results from the previous activity, and cluster the themes
that you find there.
\subsection{Advice from a Caterpillar}
\label{0ef4b185-513f-40c2-b884-6213601bbe09}
Reflect on your observations, and use them to describe your
perspective.  You might comment on aspects of your values,
professional training, and life experiences that led you to make the
observations you did, as well as the direct circumstances that
contributed to shaping your experience.
\subsection{Problem indentification}
\label{sec:org769d413}

\subsubsection{Instructions}
\label{sec:org6cc932b}
Working together with the small group, talk about any problems you
noticed.  How does the map represent stressful or concerning
experiences?  What are some alternative histories or alternative
futures that would describe how the circumstances would have changed?
\subsection{Dimension analysis}
\label{sec:org67443d9}

\subsubsection{Instructions}
\label{sec:orgcf1fd4c}

Coming back together with the full group, arrange the maps you created
across a set of dimensions.  Two dimensions would be traditional:
creating a 2-by-2 grid with ``best'' in the upper right, ``worst'' in
the lower left, and so on — but feel free to use as many dimensions as
you wish.  For example, it could be helpful to use the \href{https://en.wikipedia.org/wiki/Theory\_of\_Basic\_Human\_Values}{Theory of Basic
Human Values} to organise the scenarios.
\subsection{Build Scenarios}
\label{sec:orgf3992ba}

\subsubsection{Instructions}
\label{sec:orgcf8e046}

Working together with the full group, use the dimensions you created
in the previous activities (together with the maps and stories) to
give descriptive names to some scenarios for the future.  These should
sum up the map(s) in each quadrant (or more generally, segment) from
the diagrammatic analysis.
\subsection{Dérive Comix Part 2}
\label{f447153f-7ff5-449d-bb08-67f579dda53f}
Explore your scenarios together in your imagination and discuss what
you find there.  What are some of the things you observe from the
perspective of your new role?  What things that you observe from the
perspective of your prior training and experience?
\subsection{Connections from Kafka}
\label{34be214c-5885-4794-b93c-84e49ddad18b}
The \hyperref[a0796d9e-664b-46fa-bb37-7f6a6fc15584]{Kaijū Communicator} should now develop and communicate significant
purturbations to the scenario.
\subsection{A path forward}
\label{7c0dce3b-d5ea-4712-a771-6ff26f143686}
Work to develop a story of the future evolution to the scenario,
taking into account the meaning threats.
\subsection{Back to reality}
\label{e38d2006-bcf7-494b-bd51-d8932b1ed0cd}
As the process of building \hyperref[7c0dce3b-d5ea-4712-a771-6ff26f143686]{A path forward} develops, the \hyperref[5826c7d9-8962-433d-83c5-27a5196908ea]{Analyst} should
build a tableau of 4 meaningful symbols indexed to the four CLA
layers, summarising the exploration above.
\subsection{New patterns}
\label{ed238393-a7e4-4a0d-9eb2-3d6ab745c170}
As the process of building \hyperref[7c0dce3b-d5ea-4712-a771-6ff26f143686]{A path forward} develops, the \hyperref[48a1d6a3-800d-46bd-8a4a-0d3414ecf150]{Designer}
should write down some new design patterns that relate to the skills
of participants.
\subsection{Project Action Preview}
\label{baa168fb-37a0-4144-ab16-d4962728ea9c}
As the process of building \hyperref[7c0dce3b-d5ea-4712-a771-6ff26f143686]{A path forward} develops, the \hyperref[57d46961-a056-435e-85d2-27ab6e0de7f6]{Historian}
should write down next steps for participants to take after the
workshop.  These actions might help people learn the skills they need
to bring about any beneficial aspects of the scenario (e.g., to
prepare for an adaptive response to a challenging situation).  The
actions may need to be scaffolded by new tools, policies, or other
innovations: write these down, also.
\subsection{Repeat Phase II}
\label{092e4fe4-ee4f-494d-8776-c5f1389e8dc0}
Reform groups, and run the exercises from Myths with a new
\hyperref[57d46961-a056-435e-85d2-27ab6e0de7f6]{Historian}. The new Historian should recap key points from the PAR from
the previous group’s Systems-to-Litany exercise, and the team should
then explore the new scenario, following all of the steps again.

This process can be repeated more than once as time allows. As you
work through this activity, feel free to introduce connections with
the previous scenario(s) you already explored, although the new
Historian won’t be familiar with them.
\subsection{Share back}
\label{848c8c3d-cde3-48b4-9dae-23eca4db440d}
We’re at the end of our time together, let’s share back any crucial points with a full-group PAR.

\section{Roadmap}
\label{92e18906-d0e6-4e73-a9cf-fbdad931f3cf}
\subsection{Summary}
\label{sec:orga4e8af5}

\textbf{*Context} a group needs to coordinate its activities over a period of
time. \textbf{If} the landscape is complex and not completely knowable BUT
adjustment to action based on feedback is possible; \textbf{Then} use an
explicit mechanism to share information about goals, obstacles, methods,
and resources.

\subsection{Instructions}
\label{sec:org7206ec0}

\subsubsection{Overall process}
\label{sec:org0b7c145}

\begin{itemize}
\item \textbf{Review Phase I and Phase II}:
\begin{itemize}
\item \hyperref[f5a1bc15-5abb-44d6-8f7a-e254974c9002]{Project Action Review}
\end{itemize}
\item \textbf{Repeat with variations, as needed.}
\item \textbf{Respond to participants in the pilot workshops, and keep building energy}
\item \textbf{Carry out meta-review periodically:}
\begin{itemize}
\item \hyperref[56ce8d31-d3d6-4493-bb41-b07d810afbcc]{Causal Layered Analysis}
\end{itemize}
\item Adjust workshop methods based on what we learned
\item \textbf{Develop supportive methods and technology to scaffold ongoing work by communities}
\begin{itemize}
\item\relax [BROKEN LINK: 2b1ca06d-486e-4398-a2c9-a4a9e303eaa3]
\end{itemize}
\item ``Dogfood'' this process by developing methods and technology that we ourselves find useful
\end{itemize}

\begin{enumerate}
\item Specific actions arising
\label{sec:org5d19d65}

\begin{itemize}
\item[{$\square$}] Alana from Bristol pilot wants to follow up; others may as well!
\item[{$\square$}] Worth dropping a line to attendees of the Anticipation 2022 where they can be traced to say ‘thank you’
\end{itemize}
\end{enumerate}

\subsubsection{Materials}
\label{sec:org550cc26}

\begin{itemize}
\item Shared experience of facilitators
\item This archive of working methods
\end{itemize}

\begin{enumerate}
\item Previous Runs of This or Related Workshops
\label{sec:org18dfa67}
\begin{enumerate}
\item 2022 November 18, Anticipation 2022 ``Going Meta''
\label{sec:org1bc1a2a}
\item 2022 November 3rd pilot in Bristol on Public Space and Public Health, Arnolfini Arts
\label{sec:org79eccc8}
\item 2021 PLoP
\label{sec:org218ec5b}
\item 2021 Connected Learning Summit
\label{sec:orgdec1f98}
\item 2021 Oxford Brookes University Creative Industry Festival
\label{sec:orgeb83126}
\item 2019 Anticipation Conference in Oslo “Fictional Peeragogical Anticipatory Learning Exploration”
\label{sec:orgc8de25e}
\item 2014 OpenEdJam with ‘zine \url{http://is.gd/openedjam}
\label{sec:org2785ca5}
\end{enumerate}
\end{enumerate}


\subsubsection{Intended outcome}
\label{sec:org0f11016}
We want to help communities grow and solve complicated problems that they would not readily solve without the kinds of interventions we’re offering.

\subsection{Phase III Feedback}
\label{sec:orgda953eb}

\subsubsection{Let’s make and maintain our own roadmap}
\label{sec:org7862d11}

\begin{itemize}
\item Recognise that we are living inside ``Phase III''
\item Make sure that we have things well prepared
\item We do have a \texttt{yasnippet} based workflow for running the \textbf{meetings}, could we do something similar with the workflow for the \textbf{workshop}?
\item Use a common agenda file for the Abby project; it will live somewhere we can all access \& track tasks.
\begin{itemize}
\item A potential implementation is at \texttt{\textasciitilde{}/exp2exp.github.io/src/erg/agenda.org} — that’s worth a look as a starter pack.
\item However, we could do something similar in a more distributed way, e.g., the following will create an agenda for all \texttt{TODO} items files within this repo that have the :WS: file tag.  There are only two such items so far.
\item We could, similarly, make additional mixes for different managerial views into the repo.
\end{itemize}
\end{itemize}

\begin{verbatim}
(defun org-scrum-board (&optional filter)
  (interactive)
  (let ((org-todo-keywords
         '((sequence "TODO(t)" "|" "DONE(y)"))))
  (org-todo-list "TODO")
  (when filter
    (org-agenda-filter-apply `(,filter) 'regexp))))

(defun org-scrum-board-workshop ()
  (interactive)
  (let ((org-agenda-sorting-strategy '((todo todo-state-up category-keep tag-up)))
        (org-agenda-files '("~/exp2exp.github.io/src/"))
        (org-agenda-title "Workshop Roadmap"))
    (org-scrum-board ":WS:")))

(global-set-key (kbd "C-c R e") 'org-scrum-board-workshop)
\end{verbatim}

\subsubsection{Let’s make and maintain our overall taxonomy}
\label{sec:org243fcde}

\begin{itemize}
\item Terms like “scenario” and “role” aren’t entirely clear, much less specifics like “Kaijū Communicator”: can we clear this up with a taxonomy?
\item We’ll need to add some additional terms (like “disorder” or “meta”) per workshop run, but good to have a start with the core lexicon, then we could see how the other terms relate
\end{itemize}

\subsubsection{Phase III Evolution}
\label{sec:org48d00fb}

\begin{itemize}
\item Having refactored things as a collection of Org Roam files, and reexported them as a linear document, we’re starting to have a ``map'' of the domain.
\item Demo task tracker, above.  But, are these the right set of tasks?
\end{itemize}
\subsection{Project Action Review}
\label{f5a1bc15-5abb-44d6-8f7a-e254974c9002}
\subsubsection{Summary}
\label{sec:orgf126cff}

\textbf{*Context} Work in progress. \textbf{If} we are working on something together
BUT we might lose momentum; \textbf{Then} use a review template to think about
our progress. Questions like the following can be asked at any point in
a project, and provide a momentary record of perspectives which can be
analysed later.

\begin{enumerate}
\item \emph{Review the intention: what do we (did we) expect to learn or make
together?}

\item \emph{Establish what is happening: what and how are we learning?}

\item \emph{What are some different perspectives on what's happening?}

\item \emph{What did we learn or change?}

\item \emph{What else should we change going forward?}
\end{enumerate}
\subsection{Causal Layered Analysis}
\label{56ce8d31-d3d6-4493-bb41-b07d810afbcc}
\textbf{*Context} Work in progress. \textbf{If} we are working on something together BUT
we might lose direction; \textbf{Then} review our previous \hyperref[f5a1bc15-5abb-44d6-8f7a-e254974c9002]{Project Action
Review} data to sense-make about our progress and intentions.  This
process can be carried out routinely, e.g., after 6 or more sessions.
The template suggested by Causal Layered Analysis theory can be used.
This can be adapted in light of our roles, to help formulate new
patterns.

\begin{enumerate}
\item Litany (However\ldots{})
\item System (Because\ldots{})
\item Worldview (Therefore\ldots{})
\item Myth (Specifically\ldots{})
\end{enumerate}

\subsubsection{Examples}
\label{sec:org4fb1c0d}

The Emacs Research Group developed a set of CLAs in 2021.

\begin{itemize}
\item\relax [BROKEN LINK: eba531ea-7a47-4dba-bdd5-045d27cf0033]
\item\relax [BROKEN LINK: ef397d5d-b0d5-4764-b0f3-b1fb9f240302]
\item\relax [BROKEN LINK: 8cfb334a-4176-4fa2-ac2f-8efff5f3c842]
\item\relax [BROKEN LINK: 4b759839-5721-41e8-bce7-04606183bfc9]
\item\relax [BROKEN LINK: 732219c0-9784-4593-b781-b82e54e948ce]
\end{itemize}

However, we left off at that point; see feedback below.

\subsubsection{Feedback}
\label{sec:orgec3c179}

\begin{itemize}
\item After discussion it seemed that there wasn't sufficient group buy-in to the CLA-creation process in these early exercises.  Although we had appropriated the PAR, we hadn't appropriated CLA.  Would it be possible to revisit the CLA to make it more meaningful?
\item Addressing this could help make “Phase III” more useful in general!
\end{itemize}
\end{document}
